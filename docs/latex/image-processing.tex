\subsection*{Applying a Filter to Your own Image}

To use the command-\/line Image Processing tool on your own images, you first need to obtain the source code for this project. Please view \hyperlink{getting-started}{Getting Started} on how to obtain a local copy of the source code to get started. \hyperlink{getting-started}{Getting Started} has a section titled \textquotesingle{}Running an Image Filter on An Image (Command-\/\+Line Tool)\textquotesingle{} that describes how to compile and run the \hyperlink{classImage}{Image} Processing command-\/line tool.

Once you have obtained a local copy of the source code on your local machine you are ready to run your own image filtering.
\begin{DoxyEnumerate}
\item Start run the Docker container using the instructions in the {\bfseries Build Docker Environment} section in \hyperlink{getting-started}{Getting Started}.
\item Within your Docker container, navigate into the {\ttfamily project/\+Image\+Processing/} directory where the source code is located. This can be done by running the {\ttfamily cd project/\+Image\+Processing/} command.
\item Run the command {\ttfamily make clean \&\& make -\/j} to compile all of the source code required to run the image processor.
\item You are now ready to run the project using the following command\+: {\ttfamily ./image\+\_\+processor $<$input\+Image$>$ $<$filter$>$ $<$output\+Image$>$}
\begin{DoxyEnumerate}
\item {\ttfamily $<$input\+Image$>$} is a string that represents the relative path to the image that you want to apply a filter to (e.\+g. {\ttfamily \char`\"{}data/dog.\+png\char`\"{}})
\item {\ttfamily $<$filter$>$} is a string that represents the type of filter that you want to apply to your image. The filter options include\+:
\begin{DoxyEnumerate}
\item greyscale -\/ Applies a Greyscale filter to the {\ttfamily $<$input\+Image$>$} specified.
\item threshold -\/ Applies a Threshold filter with a threshold of 0.\+50 to the {\ttfamily $<$input\+Image$>$} specified.
\item threshold-\/low -\/ Applies a Threshold filter with a threshold of 0.\+25 to the {\ttfamily $<$input\+Image$>$} specified.
\item threshold-\/high -\/ Applies a Threshold filter with a threshold of 0.\+75 to the {\ttfamily $<$input\+Image$>$} specified.
\item mean\+\_\+blur -\/ Applies a Mean Blur filter to the {\ttfamily $<$input\+Image$>$} specified.
\item gaussian -\/ Applies a Gaussian Blur filter with a kernel size of 5 and a sigma value of 2 to the {\ttfamily $<$input\+Image$>$} specified.
\item double\+\_\+threshold\+\_\+filter -\/ Applies a Double Threshold filter with a low threshold ratio of 0.\+10 and a high threshold ratio of 0.\+20 to the {\ttfamily $<$input\+Image$>$} specified.
\item hysteresis\+\_\+filter -\/ Applies a Hysteresis filter with a weak threshold of 25.\+0 and a strong threshold of 255.\+0 to the {\ttfamily $<$input\+Image$>$} specified.
\item canny-\/edge-\/detect -\/ Applies a Canny Edge Detection to the {\ttfamily $<$input\+Image$>$} specified.
\end{DoxyEnumerate}
\item {\ttfamily $<$output\+Image$>$} is a string that represents the relative path and file name of the image that will be generated by applying the specificed filter.
\end{DoxyEnumerate}
\end{DoxyEnumerate}

\subsection*{Adding Your Own Filters}

Adding your own filters is relatively easy with this project.
\begin{DoxyEnumerate}
\item Determine if the filter that you want to add fits into the category of a \hyperlink{classSimpleFilter}{Simple\+Filter}, \hyperlink{classRelatedPixelFilter}{Related\+Pixel\+Filter}, or \hyperlink{classConvolutionFilter}{Convolution\+Filter} based on the definitions above.
\begin{DoxyEnumerate}
\item If so, create a new class for your filter that is derived from the correct categoral class (\hyperlink{classSimpleFilter}{Simple\+Filter}, \hyperlink{classRelatedPixelFilter}{Related\+Pixel\+Filter}, or \hyperlink{classConvolutionFilter}{Convolution\+Filter}).
\item If not, create a new categorical class for your new type of filter. This categorical class should be derived from the base \hyperlink{classFilter}{Filter} class.
\begin{DoxyEnumerate}
\item This new categorical class will need to implement the pure virtual {\ttfamily Apply} method as defined in the \hyperlink{classFilter}{Filter} class. Use the \hyperlink{classSimpleFilter}{Simple\+Filter}, Related\+Pixels\+Filter, or \hyperlink{classConvolutionFilter}{Convolution\+Filter} classes as templates for achieving this.
\end{DoxyEnumerate}
\end{DoxyEnumerate}
\item Once your have a categorical class that is derived from \hyperlink{classFilter}{Filter} that is appropriate for your new filter, create a new derived class from the correct categorical class for your brand new filter. Your new class will need to implement the pure virtual {\ttfamily Apply\+At\+Pixel} method as defined in your categorical class. This method describes what the image filter does to each pixel in the image to apply the filter. Use the \hyperlink{classSimpleFilter}{Simple\+Filter}, Related\+Pixels\+Filter, or \hyperlink{classConvolutionFilter}{Convolution\+Filter} classes as templates for achieving this.
\item Within {\ttfamily Image\+Processor.\+h} include the header file for your newly defined filter class and add the filter to the map of strings and filters within the \hyperlink{classImageProcessor}{Image\+Processor} constructor. Use the already defined filters as a template.
\item Re-\/compile the source code using {\ttfamily make} as described in the {\bfseries Running an Image Filter on An Image} section above.
\item Start applying your new filter to images!
\end{DoxyEnumerate}

\subsection*{Image Processor Design Structure}

All of the classes for this image processor iteration were created with the goal of designing code that is extensible, flexible and non-\/repetitive. Polymorphism, encapsulation and inheritance were used to achieve those goals. The filter class was implemented to serve as the base for all derived filter classes. Categorial filter classes, such as \hyperlink{classRelatedPixelFilter}{Related\+Pixel\+Filter}, were created to categorize filters based on how they apply a filter to an image. This section aims to give a detailed description of the design implemented.

Our design follows a facade design pattern, the complexity of the system is hidden and the \hyperlink{classImageProcessor}{Image\+Processor} behaves as the interface for a user to interact with. The Image Processor contains the possible filters that can be applied to an image, such as {\ttfamily greyscale} and {\ttfamily canny-\/edge-\/detect}, and has a dependency to \hyperlink{classImage}{Image}, which means that an \hyperlink{classImage}{Image} is not held by the \hyperlink{classImageProcessor}{Image\+Processor} but rather the \hyperlink{classImageProcessor}{Image\+Processor} interacts with the \hyperlink{classImage}{Image} class through the {\ttfamily Apply\+Filer(...)} method based on any \hyperlink{classFilter}{Filter} type passed as an input parameter. For this iteration, we assumed that the {\ttfamily sobel} and {\ttfamily non-\/max} filters were only used if implementing the canny edge detection, therefore the Image Processor cannot directly implement these two filters. Moreover, it was assumed that any image passed to the Image Processor as input existed.



Specifically, the design chosen to represent an image from the filesystem was as an instance of the \hyperlink{classImage}{Image} class, in which each \hyperlink{classColor}{Color} object represents a single pixel in the image, so an \hyperlink{classImage}{Image} object contains a unique pointer to an array of Colors.



\hyperlink{classFilter}{Filter} Class serves as the abstract base class from which 3 main categories are derived\+: \hyperlink{classSimpleFilter}{Simple\+Filter}, \hyperlink{classRelatedPixelFilter}{Related\+Pixel\+Filter} and \hyperlink{classCanny}{Canny}.



The \hyperlink{classSimpleFilter}{Simple\+Filter} class is for filters that change a single image pixel without information about any other pixel in the image. Therefore, \hyperlink{classSimpleFilter}{Simple\+Filter} serves as the abstract class for greyscale, double threshold, and threshold image filters.



Additionally, the \hyperlink{classRelatedPixelFilter}{Related\+Pixel\+Filter} class is an abstract class for any filter that calculates the new, filtered pixel color components based on other pixels in the image. Two filters that are derived from this class and that have been implemented are the \hyperlink{classHysteresisFilter}{Hysteresis\+Filter} and the \hyperlink{classNonMaxSupFilter}{Non\+Max\+Sup\+Filter} to apply a hysteresis and non-\/maximum suppression filter to an image, respectively. Moreover, \hyperlink{classConvolutionFilter}{Convolution\+Filter} is a subcategory of the \hyperlink{classRelatedPixelFilter}{Related\+Pixel\+Filter} abstract class since a convolution filter needs to take into account surrounding pixels to determine the filtered value of a pixel and a kernel to be applied to those surrounding pixels.



Each instance of the \hyperlink{classConvolutionFilter}{Convolution\+Filter} class has a vector of kernels. \hyperlink{classKernel}{Kernel} is a class that stores the matrix representing the kernel values that need to be applied to the a pixel and its surrounding pixels to determine the new filtered pixel color values. This abstract \hyperlink{classConvolutionFilter}{Convolution\+Filter} class serves as the basis for the \hyperlink{classGaussianFilter}{Gaussian\+Filter} and the \hyperlink{classSobelFilter}{Sobel\+Filter}, which called the parent constructor to initialize their respective kernels.



The last category of filters is the \hyperlink{classCanny}{Canny} filter, which implements the canny edge detection algorithm through the use of a composite pattern.



As mentioned above, all of the classes for this image processor iteration were created with the goal of designing code that is extensible, flexible and non-\/repetitive, thereby allowing for all of our code to be close to modification and open to extension. Adding a new filter is a straightforward process. First, determine if the filter that you want to add fits into the category of a \hyperlink{classSimpleFilter}{Simple\+Filter}, \hyperlink{classRelatedPixelFilter}{Related\+Pixel\+Filter}, or \hyperlink{classConvolutionFilter}{Convolution\+Filter} based on the definitions above. If it follows one of those definitions, create a new class for your filter that is derived from the correct categorical class (\hyperlink{classSimpleFilter}{Simple\+Filter}, \hyperlink{classRelatedPixelFilter}{Related\+Pixel\+Filter}, or \hyperlink{classConvolutionFilter}{Convolution\+Filter}). Your new \hyperlink{classFilter}{Filter} class will need to implement the {\ttfamily Apply\+At\+Pixel(...)} method, which applies the given filter to a specific pixel (x,y). On the other hand, if it is a “new” categorical filterm create a class for your filter that inherits directly from \hyperlink{classFilter}{Filter} class. Note that this new categorical filter will need to implement the pure virtual {\ttfamily Apply(...)} method as defined in the \hyperlink{classFilter}{Filter} class.

Note, that although the explanation above uses our implementation of filters, the \hyperlink{classRescueSimulation}{Rescue\+Simulation} uses the {\ttfamily opencv} library by creating an \hyperlink{classObjectDetector}{Object\+Detector} that implements the canny and edge detect filter via {\ttfamily opencv} directly to check if the robot was found. The reason behind this design choice was primarily speed. Our implementation of canny edge detection took approximately 15 seconds, while the {\ttfamily opencv} library canny edge detection took less than 1 second. Moreover, we created an \hyperlink{classObjectDetector}{Object\+Detector} class rather than an image processor because we also needed to determine which pixel in the image the robot was found at.

\subsection*{Copyright}

\copyright{} 2021 Laura Arias Fernandez, Michael Weiner, and Malik Khadar All rights reserved. 