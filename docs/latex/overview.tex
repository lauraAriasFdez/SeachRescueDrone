This project consists of 2 main subsystems\+: \hyperlink{classImage}{Image} processing and the Rescue simulation

{\itshape {\bfseries 1. Image Processing\+:}} subsystem in charge of applying image filters on images stored locally on your machine. Specifically, the subsystem is designed to implement a Canny Edge detection, yet it also allows for more image filtering extensibility if the user desires to add more filters.

Our design follows a facade design pattern, the complexity of the system is hidden and the \hyperlink{classImageProcessor}{Image\+Processor} behaves as the interface for a user to interact with. The Image Processor contains the possible filters that can be applied to an image, such as {\ttfamily greyscale} and {\ttfamily canny-\/edge-\/detect}, and has a dependency to \hyperlink{classImage}{Image}, which means that an \hyperlink{classImage}{Image} is not held by the \hyperlink{classImageProcessor}{Image\+Processor} but rather the \hyperlink{classImageProcessor}{Image\+Processor} interacts with the \hyperlink{classImage}{Image} class through the Apply\+Filer method based on any \hyperlink{classFilter}{Filter} type passed as an input parameter. The {\ttfamily sobel} and {\ttfamily non-\/max} filters were only designed as if they were being used as part of the canny edge detection, therefore the \hyperlink{classImage}{Image} Processor cannot directly implement these two filters. Moreover, it was assumed that any image passed to the \hyperlink{classImage}{Image} Processor as input existed on the filesystem.



Images read in from the filesystem are represented as an instance of the \hyperlink{classImage}{Image} class, in which each \hyperlink{classColor}{Color} object represents a single pixel in the \hyperlink{classImage}{Image}, thereby an \hyperlink{classImage}{Image} contains a unique pointer to an array of Colors.



The \hyperlink{classFilter}{Filter} class serves as the abstract base class from which 3 main categories are derived\+: \hyperlink{classSimpleFilter}{Simple\+Filter}, \hyperlink{classRelatedPixelFilter}{Related\+Pixel\+Filter} and \hyperlink{classCanny}{Canny}.



\hyperlink{classSimpleFilter}{Simple\+Filter} has several derived classes\+:
\begin{DoxyItemize}
\item \hyperlink{classDoubleThresholdFilter}{Double\+Threshold\+Filter} conducts a Double Threshold filter on an image.
\item \hyperlink{classGreyScaleFilter}{Grey\+Scale\+Filter} conducts a Grescyale filter on an image.
\item \hyperlink{classThresholdFilter}{Threshold\+Filter} conduts a Threshold filter on an image.
\end{DoxyItemize}

\hyperlink{classRelatedPixelFilter}{Related\+Pixel\+Filter} has several derived classes\+:
\begin{DoxyItemize}
\item \hyperlink{classConvolutionFilter}{Convolution\+Filter} is an abstract, base class that represents another sub-\/category of filters that require neighboring pixels in an image {\bfseries and} a kernel to create the new image with the filer applied.
\item \hyperlink{classHysteresisFilter}{Hysteresis\+Filter} conducts a Hysteresis filter on an image.
\item \hyperlink{classNonMaxSupFilter}{Non\+Max\+Sup\+Filter} conducts a Non-\/\+Max Suppression filter on an image.
\end{DoxyItemize}

\hyperlink{classConvolutionFilter}{Convolution\+Filter} has several derived classes\+:
\begin{DoxyItemize}
\item \hyperlink{classGaussianFilter}{Gaussian\+Filter} conducts a Gaussian filter on an image.
\item \hyperlink{classMeanBlurFilter}{Mean\+Blur\+Filter} conducts a Mean Blur filter on an image.
\item \hyperlink{classSobelFilter}{Sobel\+Filter} conducts a Sobel filter on an image.
\end{DoxyItemize}

The \hyperlink{classKernel}{Kernel} class is used by the \hyperlink{classConvolutionFilter}{Convolution\+Filter} class, and its derived classes, to store and manipulate the kernels for calculation to apply a filter to an image.

{\itshape {\bfseries 2. Rescue Simulation\+:}} subsystem that consists of creating a simulation in which a drone searches for a robot in a scene. This subsystem uses a web-\/based visualization system provided by the C\+S\+C\+I3081 staff.

In the simulation there are multiple patterns that were used in order to build an extensible simulation. First, the facade design pattern was implemented in order to allow the complexity of the system to be hidden behind a class called \hyperlink{classRescueSimulation}{Rescue\+Simulation}.



In order to build the Entities, we decided to use an inheritance hierarchy as that will allow for entities to share common attributes like a position or an entity\+ID (to name a few).



The Factory pattern was used to create Entities (robot, drone, hospital...etc). No entity was created in the simulation directly, rather the simulation contains a composite factory that when it receives the command to create an entity it passes the {\ttfamily picsonjson} command to the {\ttfamily Create\+Entity(...)} method and the appropriate factory class handles the correct entity creation.



Additionally,the Strategy pattern was implemented for the \hyperlink{classDrone}{Drone} movement, thereby the \hyperlink{classDrone}{Drone} will rely on a member variable that points to an instance of the \hyperlink{classIMovement}{I\+Movement} class to generate the set of 3D vectors the drone must move to depending on its status\+: looking for the robot and going to the robot will correspond to patrol and beeline movement respectively. This delegation of movement is necessary as this will allow for other objects to be added to the “movable” property more easily, if the program was to be extended.



Finally, in order to allow for searching the drone we created an \hyperlink{classObjectDetector}{Object\+Detector} class that uses the open-\/source {\ttfamily opencv} library, rather than our Image Processing subsystem, to perform blob detection and canny edge detect as that ended up being a much faster approach than using our filters, which took 15 sec to process a single image.



\subsection*{Copyright}

\copyright{} 2021 Laura Arias Fernandez, Michael Weiner, and Malik Khadar All rights reserved. 